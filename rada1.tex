
\documentclass {article}

\makeatletter
\makeatother
\usepackage[czech]{babel}
\usepackage[utf8]{inputenc}

\usepackage[letterpaper]{geometry}
\usepackage{graphicx}

\usepackage{csquotes}
\geometry{verbose,tmargin=1.5cm,bmargin=2cm,lmargin=2cm,rmargin=2cm}
\marginparwidth 0pt

\pagenumbering{gobble}
\begin{document}

\pagestyle{empty}
\author {psal vykač, formulace volně modifikované}

\title{První porada}
\date{6. 10. 2018, Klášter svaté dobrotivé}
\maketitle

%\vspace{-50px}

\section{Co chceme?} % (fold)
\label{sec:co_chceme_}
	\begin{itemize}
	\item -má to mít smysl
	\item máme to umět vysvtlit dětem
	\item Má to být dynamické
	\item má to být motivovatelné
	\end{itemize}
% section co_chceme_ (end)

\section{Navázání} % (fold)
\label{sec:navazani}
	\begin{itemize}
		\item	Koala: Challenge super, motivuje
		\item	Kocour: přijde mu blbost vázat slib na stezku
		\item	Alvis: není rozumné podmiňovat slib
		\item	Motivační věci obecně podporovány
		\begin{itemize}
			\item	slib kontroverzní
			\item	Může být dobrá motivace
			\item	Pískle: Nesouvisí to spolu.
			\item	Šerif: to nevadí, je to odměna a je fajn, že je viditelná
			\item	Humr: Chce obnovit bejčky, s indikací postupu ve stezce
			\item	Pískle: lépe by se dětem vysvětlovaly podmíněnosti slibů, i kdyby nebyly ostře podmíněny pravilem
			\item Humr: mohli bychom debaty o slibování zahajovat dřív, než na táboře
			
		\end{itemize}
		
		\item Nechceme mít splněnost stezek povinnou. Nechcem z toho testy.
		\item Ostatní: (souhlasně pokyvují) To je moudrý muž.
		\item Pískle: Takže můžem dát všem jedničky
		\item Humr: To byl můj point



	\end{itemize}
% section navázání (end)

\section{Rozložení} % (fold)
\label{sec:rozlozeni}
	
	Nováček na rok,
	Potom po dvou letech konzistentně
	\subsection{Startovní rozložení} % (fold)
	\label{sub:start}
		\begin{itemize}
			\item Koaliny holky: nováček
			\item Kompoti: nováček
			\item Urzoni: 1. level asi
			\item Eskymáci: Ještě k  dořešení, možná indivuduálně
		\end{itemize}
	% subsection startovní_rozložení (end)



% section rozložneí (end)

\section{Obsahy a přidružené} % (fold)
\label{sec:obsahy}
	\begin{itemize}
		\item Chceme mít rozdělené úkoly do kategorií (zdastnost, zdravověda, zručnost...)
		\item Chceme mít v klubovně vyvěšenou tabulku ůspěchů (teeppe of fame) \footnote{Na konci roku může být rituál.}
		\item
	
	\subsection{Chceme mít kategorie fixní přes levely} % (fold)
	\label{sub:chceme_kategorie_fixni_pres_levely}
	\begin{itemize}
		\item Co třeba zdatnost? Chceme mít vůbec tuto kategorii ve starších?
		\item Co třeba změnou stylu (udělám 50 kliků $\rightarrow$ hecnu se chdit každé ráno běhat)
		\item Chcem hodnotit výkony? 

	\end{itemize}
	% subsection chceme_mít_kategorie_fixní_přes_levely (end)
		\item Nechceme úkoly splnitelné temporárním nabiflováním (Seton a Baden-Poval)
		\item Co podpisy kamarádů? Co úkoly na poctivost?
		\item Alvis: Chteli jsme měřitelné a uchopitelné úkoly.
		\item Pískle: Jsem proti podpisů kamarádů a rodiny (všichni přikivují)
		\end{itemize}

% section obsahy (end)

\section{Kategorie} % (fold)
\label{sec:kategorie}
	Zvolili jsme:
	\begin{itemize}
		\item Zdatnost
		\item Zdravověda
		\item uzly
		\item Interakce s okolím
		\item Myšlenkové základy skautingu
		\item Příroda / Přírodověda
		\item Topografie
		\item Tvořivost (vaření)
		\item Šifry
		\item Etiketa
		\item ... \footnote{Seznam byl sepsán za zmatených podmínek a není obecně prodiskutovaný a nejspíš není kompletní. Brát jako inspraci, ne jako usnesení.} 
	\end{itemize}
% section kategorie (end)
\section{Co teď} % (fold)
\label{sec:co_ted}
	Každý jsme si vybrali jeden z aktuálně hořících levelů (nováček/1. level). Do 21. 10. 2018 máme čas na brianstorming. Ještě se ozvem.
% section co_teď (end)
\end{document}

